\chapter{Literature review}

\label{ch:lit_review}

\section{Literature related to the topic/to the problems posed in the thesis}

Przeglad literatury

Podsekcje dla kazdego z 5 problemow

1.Metody dekompozycji sygnału na składowe o prostszej strukturze\\
-Krótkookresowa transformacja Fouriera (STFT)\\
-Inne transformacje (t. falkowa, dekompozycja empiryczna)\\
2.Metody poszukiwania optymalnego pasma częstotliwościowego (składowe deterministyczne + sygnał impulsowy)\\
-Selektory bazujące na momentach, kwantylach lub innych właściwościach sygnału\\
3.Filtracja sygnału na podstawie selektorów
-Ustalenie granicznych poziomów selektorów\\
4.Modelowanie i filtracja sygnału na podstawie modelu autoregresyjnego (AR)\\
-Metody estymacji parametrów modeli wysokiego rzędu z szumem gaussowskim i niegaussowskim\\
-Stabilność dopasowanego modelu\\
5.Modelowanie sygnału drganiowego w zmiennych warunkach eksploatacyjnych\\
-Model AR o okresowo zmiennych współczynnikach (PAR z długim okresem zmienności)

\section{Discussion}

Dyskusja nad roznymi podejsciami, wady i zalety metod z literatury. Cos jak dyskusja nad roznymi metodami ustalenia poziomow granicznych selektorow.

\section{Conclusions}
Jak mozna poprawic (jak to poprawilemw pracy) metody z literatury.