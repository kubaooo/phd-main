\chapter{Introduction}
Szkic troche na podstawie http://kipworldblog.blogspot.com/2014/05/a-brief-outline-for-organisingwriting.html
\section{Motivation and Objectives}

Motywacja do detekcji uszkodzen (elementow) maszyn gorniczych w ogole: Awarie, przestoje, podwyzszone narazenie na zbyt wysokie natezenie dzwieku na stanowisku pracy, niebezpieczne zdarzenia (dymy, pozary).
Cele: wykrywanie uszkodzen (elementow) maszyn gorniczych we wczesnym stadium rozwoju uszkodzenia, aby zapobiec awariom i niebezpiecznym zdarzeniom, ograniczyc przestoje, dac mozliwosc na zaplanowanie remontow w odpowiednim czasie.

Alternatywnie:
Motywacja do wykorzystania metod stochastycznych w analizie sygnalow drganiowych:
Drgania maszyn maja charakter losowy ze wzgledu na...(nierownosci powierzchni elementow maszyn, tj. biezni i el. tocznych lozysk, kol zebatych).
Ponadto, wplyw na postac sygnalu maja zrodla zewnetrzne (czujniki nie zbieraja jedynie drgan zwiazanych ze wzajemnym kotaktem kol zebatych czy biezni i elementow tocznych - kazdy kontakt z obudowa maszyny jest potencjalnym zrodlem dodatkowych skladowych sygnalu drganiowego - zaklocen ).
Rowniez inne maszyny pracujace w poblizu sa potencjalnym zrodlem drgan zmierzonych na maszynie, ktorej elementy sa diagnozowane.
W przypadku lozysk - losowosc spowodowana przez jitter (lozysko\_b - stad local maxima a nie cyclostationarity)
W niektorych maszynach mamy do czynienia takze z losowo zmiennym obciazeniem (koparka kolowa).
W przypadku wielu uszkodzen wystepujacych w jednej maszynie mozna oczekiwac roznego charakteru sygnalow zwiazanych z tymi uszkodzeniami (deterministyczny, losowy).
Cele: wykorzystac narzedzia oparte na metodach stochastycznych w celu uzyskania informacji o uszkodzeniu lokalnym, na podstawie sygnalow zmierzonych na maszynach gorniczych w warunkach pracy kopalni.

\section{Research Problem and Hypotheses}

Tu po kolei wypunktuje problemy, ktore napotkalem w realizacji pracy (zaklocenia od maszyn pracujacych w poblizu - local maxima + selektory + filtrowanie z selektorow, przypadkowe impulsy - selektory + filtrowanie, jitter - local maxima + selektory (impulsowosc zamiast cyklicznosci), zmiana obciazenia w koparce - PAR, dwa uszkodzenia - AR do usuniecia z sygnalu jedengo uszkodzenia + selektory z filtrowaniem do drugiego uszkodzenia)
Hipoteza - zaproponowane narzedzia pozwalaja na detekcje uszkodzen lokalnych w maszyanch gorniczych w przypadku wystepowania wyzej wymienionych problemow, w czym przewyzszaja dotychczas stosowane metody.


\section{Outline of the Thesis}

Odpowiednik "The paper is structured as follows..."


\section{Delimitations}

Inne problemy wystepujace w detekcji uszkodzen maszyn gorniczych, ktore jednakze nie beda poruszane w pracy. Inne uszkodzenia, nie-lokalne. Uszkodzenia dwoch lozysk o identycznej budowie (czestotoliwosciach charakterystycznych). Analizy wielokanalowe (i wielowymiarowe).